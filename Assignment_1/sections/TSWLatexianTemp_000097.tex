\section{Rectilinear Planar Embedding}

\subsection{2.1}

Table \ref{tab:Xvf} shows the values for {\it x_{vf}} that correspond to Figure 3 from the Assignment Document.

\begin{table}[h]
\centering
\begin{tabular}{l | l l l l l}
x_{vf} & a & b & c & d  & e  \\
\hline
v_{1}  & 0 & 1 & 1 & 0  & 0  \\
v_{2}  & 0 & 0 & 1 & 1  & 0  \\
v_{3}  & 1 & 0 & 1 & 1  & 1  \\
v_{4}  & 0 & 0 & 0 & -1 & 1  \\
v_{5}  & 1 & 0 & 0 & 0  & -1 \\
v_{6}  & 1 & 1 & 0 & 1  & 1  \\
v_{7}  & 0 & 0 & 0 & 0  &   0\\
\end{tabular}
\caption{{\it x_{vf} values for Figure 3 in Assignment Document}
\label{tab:Xvf}
\end{table}


Table \ref{tab:Zvf} shows the values for {\it {z_{fg}} that correspond to Figure 3 from the Assignment Document. There are 13 break-points in this graph.
\begin{table}[h]
\begin{tabular}{llllll}
z\_fg & a & b & c & d & e \\
a     & - & 0 & 0 & 0 & 0 \\
b     & 2 & - & 1 & 1 & 0 \\
c     & 1 & 1 & - & 0 & 0 \\
d     & 0 & 1 & 0 & - &   \\
e     & 4 & 0 & 0 & 0 & - \\

\end{tabular}
\caption{{\it z_{fg} values for Figure 3 in Assignment Document}
\label{tab:Zvf}
\end{table}

