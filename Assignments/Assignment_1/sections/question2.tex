\section{Rectilinear Planar Embedding}

\subsection{Exercise 2.1}

Table \ref{tab:Xvf} shows the values for {\(x_{vf}\)} that correspond to Figure 3 from the Assignment Document.

\begin{table}[h]
  \centering
  \begin{tabular}{l | l l l l l}
  \(x_{vf}\) & a & b & c & d  & e  \\
  \hline
  \(v_{1}\)  & 0 & 1 & 1 & 0  & 0  \\
  \(v_{2}\)  & 0 & 0 & 1 & 1  & 0  \\
  \(v_{3}\)  & 1 & 0 & 1 & 1  & 1  \\
  \(v_{4}\)  & 0 & 0 & 0 & -1 & 1  \\
  \(v_{5}\)  & 1 & 0 & 0 & 0  & -1 \\
  \(v_{6}\)  & 1 & 1 & 0 & 1  & 1  \\
  \(v_{7}\)  & 0 & 0 & 0 & 0  & 0
  \end{tabular}
  \caption{\(x_{vf}\) values for Figure 3 in Assignment Document}
  \label{tab:Xvf}
\end{table}


Table \ref{tab:Zvf} shows the values for \(z_{fg}\) that correspond to Figure 3 from the Assignment Document. There are 13 break-points in this graph.
\begin{table}[h]
  \centering
  \begin{tabular}{l|lllll}
  \(z_{fg}\) & a & b & c & d & e \\\hline
  a     & - & 0 & 0 & 0 & 0 \\
  b     & 2 & - & 1 & 1 & 0 \\
  c     & 1 & 1 & - & 0 & 0 \\
  d     & 0 & 1 & 0 & - & 1 \\
  e     & 4 & 0 & 0 & 0 & - \\
  \end{tabular}
  \caption{\(z_{fg}\) values for Figure 3 in Assignment Document}
  \label{tab:Zvf}
\end{table}

We are asked to draw a rectilinear layout of graph 2(a) from the assignment description
\begin{figure}[ht]
  \centering
  \includegraphics[scale=0.07]{images/planar-drawing.png}
  \caption{Rectilinear drawing of Graph 2(a) in assignment desc}
  \label{fig:rectidraw}
\end{figure}

\subsection{Exercise 2.2}
Let \(f_e\) be the external boundary cycle and \(F\) be the set of all boundary cycles. Given two boundary cycles \(x\) and \(y\), inner turns (from \(x\) to \(y\)) are denoted by \(z_{xy}\) and outer turns (from \(y\) to \(x\)) are denoted by \(z{yx}\).
\begin{align}
  \sum_{v} x_{vf_{e}} +&\, \sum_{b\in F \backslash {f_e}} z_{f_{e}b} - z_{bf_{e}} = -4\\
  \forall b \in F \backslash {f_e}:\sum_{v} x_{vf} +&\, \sum_{b\in F \backslash {f}} z_{fb} - z_{bf} = 4
\end{align}

\subsection{Exercise 2.4}
The objective function is given by \[min \sum_{\forall f \in G, f \neq g} z_{fg}\] given that the aim is to minimise the number of break points. The constraints that the problem is subjected to are:
\begin{align}
  &z_{fg}, z_{gf}& \geq& 0\\
  &\sum_{v} x_{vf_{e}} + \sum_{b\in F \backslash {f_e}} z_{f_{e}b} - z_{bf_{e}}& =& -4\\
  &\forall b \in F \backslash {f_e}:\sum_{v} x_{vf} + \sum_{b\in F\backslash {f}} z_{fb} - z_{bf}& =& 4\\
  & \sum_f x_{vf} & =&
  \begin{cases}  
    0 & \text{if \(v\) has degree \(2\)}\\
    2 & \text{if \(v\) has degree \(3\)}\\
    4 & \text{if \(v\) has degree \(4\)}
  \end{cases}
\end{align}
