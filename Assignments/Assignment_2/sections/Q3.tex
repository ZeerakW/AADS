\section{Bottom-\textit{k} sampling with strong universality}
\subsection{A Union Bound}
\subsubsection{Exercise 5}

We are asked to prove if ($I$) and ($II$) are both false, then so is (4). We note that $p=\frac{k}{n(1-a)}$, and  $n=|A|$.

Now we will assume the opposite of both (I) and (II) to be true:
%
\begin{description}
	\item[($I_{d}$)] The number of elements from $A$ that hash below $p$
	is greater than $k$.
	\item[($II_{d}$] The number of elements from $C$ that hash below $p$
	is less than or equal to $(1+b) p |C|$.
\end{description}
%
Firstly, ($I_{d}$) implies that all elements of $S$ hash below $p$. This we know as there are $k$
elements in $S$ and there are more than $k$ elements in $A$ that hash below
$p$.
Next, ($II_{d}$), $|C|\leq (1+b)p|C|$ and therefore also 
$|C \cap S| \leq (1+b)p|C|$. 
Given ($I_{d}$) and ($II_{d}$), we want to show that  $|C \cap S| \leq \frac{1+b}{1-a} fk$, thus proving ($I$), ($II$) by opposition.
%
\begin{align*}
	|C \cap S| 	\leq & (1+b)p|C| & \\
				=& (1+b) \frac{k}{n(1-a)} |C| & \\
				=& (1+b) \frac{k}{1-a} f  & \text{$n=|A|$ and$f=|C|/|A|$ }\\
				=& \frac{(1+b)k}{1-a} f & \\
				=& \frac{1+b}{1-a} fk & 
\end{align*}

\subsection{Upper bound with 2-independence}
\subsubsection{Exercise 6}
We want to prove $P_{(I)} = Pr \left[ X_A < k\right] \leq 1/r^2$.
We have the following expressions:
%
\begin{align}
	& Pr\left[ |X - \mu| \geq r \sqrt{\mu} \right] \leq \frac{1}{r^2} \label{eq:6:pr}\\
	k =& \mu_A(1-a) = \mu_A(1 - r \sqrt{k}) \label{eq:6:k} \\
	u_A =& E[X_A] = pn = k/(1-a) \label{eq:6:ua} 
	\end{align}
%


\begin{align*}
	 & Pr\left[X_A < k\right] & \\
	=& Pr\left[X_A < \mu_A(1-\frac{r}{\sqrt{k}})\right] & \text{ \eqref{eq:6:k}} \\
	=& Pr\left[X_A - \mu_A < -\frac{\mu_A \cdot r}{\sqrt{k}} \right] & \\
	=& Pr\left[\mu_A - X_A > \frac{\mu_A \cdot r}{\sqrt{k}} \right] & \text{Multiply both sides with -1} \\
	\leq & Pr\left[|X_A - \mu_A| > \frac{\mu_A \cdot r}{\sqrt{k}} \right] 
		& \\
	\leq & Pr\left[|X_A - \mu_A| > \frac{\mu_A \cdot r}{\sqrt{\mu_A}} \right]
		& \text{From \eqref{eq:6:ua} we know $\mu_A > k$} \\
	=& Pr\left[|X_A - \mu_A| > r \sqrt{\mu_A} \right]
		&\\
	\leq & Pr\left[|X_A - \mu_A| \geq r \sqrt{\mu_A} \right]
		& \\
	\leq & 1/r^2 & \text{\eqref{eq:6:pr} By Lemma 1}
\end{align*}

\subsubsection{Exercise 7}
We will prove $P_{II}$ in much the same was as we did $P_{I}$

\begin{align*}
	& Pr\left[ X_C > (1+\frac{r}{\sqrt{fk}}) \mu_C \right] & \\
	\leq & Pr\left[ X_C > (1+\frac{r}{\sqrt{\mu_C}}) \mu_C \right]
		& \text{$\mu_C > fk$} \\
		=& Pr\left[ X_C - \mu_C > \mu_C \cdot \frac{r}{\sqrt{\mu_C}} \right]
		& \\
	\leq& Pr\Big[ | X_C - \mu_C | > r \sqrt{\mu_C} \Big]
		& \\
	\leq& Pr\Big[ | X_C - \mu_C | \geq r \sqrt{\mu_C} \Big]
		& \\
	\leq& 1/r^2
\end{align*}

